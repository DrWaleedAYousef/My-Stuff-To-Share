\section{Conclusion}\label{sec:conclusion}
This paper aimed at training Recurrent Neural Networks (RNN) at the character level on Arabic and
English written poem to learn and recognize their meters that make poem sounding rhetoric or
phonetic when pronounced. This can be considered a step forward for language understanding,
synthesis, and style recognition. The datasets were crawled from several non technical online
sources; then cleaned, structured, and published to a repository that is made publicly available for
scientific research. To the best of our knowledge, using Machine Learning (ML) in general and Deep
Neural Networks (DNN) in particular for learning poem meters and phonetic style from written text,
along with the availability of such a dataset, is new to literature.

For the computational intensive nature and time complexity of RNN training, our network
configurations were not exhaustive to cover a very wide span of training parameter configurations
(e.g., number of layers, cell size, etc). Nevertheless, the classification accuracy obtained on the
Arabic dataset was remarkable, specially if compared to that obtained from the deterministic and
human derived rule-based algorithms available in literature. However, the door is opened to many
questions and more exploration; to list a few: how to increase the accuracy on English dataset, why diacritic
effect is not consistent, and why some meters possess low per-class accuracy.
