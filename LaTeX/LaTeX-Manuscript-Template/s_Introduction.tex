\section{Introduction}\label{sec:introduction}
\subsection{Arabic Language}\label{sec:arabic-language}
\begin{table}[!t]
  \centering
  \resizebox{\columnwidth}{!}{%
    \begin{tabular}{c c c c c c}
      \toprule
      \textbf{\small{Diacritics}}  & {\small{\textit{without}}} & {\small{\textit{fat-ha}}} & {\small{\textit{dam-ma}}} & {\small{\textit{kas-ra}}} & {\small{\textit{sukun}}} \\
      \midrule
      \textbf{\small{writing}}     & \textarabic{د}             & \textarabic{دَ}            & \textarabic{دُ}            & \textarabic{دِ}            & \textarabic{دْ}\\
      \textbf{\small{short vowel}} & ---                         & /\textit{a}/              & /\textit{u}/              & /\textit{i}/              & /\textit{no vowel}/\\
      \bottomrule
    \end{tabular}
  }
  \caption{\textit{The 4 Diacritics of Arabic Language. Transliterated names (1st row), writing
      style on example letter \textarabic{د}} (2nd row), and corresponding short pronunciation
    vowel (3rd row).}\label{arabic:diacritics_dal}
\end{table}
Arabic is the fifth most widely spoken language~\cite{Simons201720thEditionEthnologue}.  It is
written from right to left (RTL). Its alphabet consists of 28 primary letters and 8 further derived
letters from the primary ones, which makes all letters sum up to 36.  The writing system is cursive;
hence, most letters are joined and a few letters remain disjoint.

Each Arabic letter represents a consonant, which means that short vowels are not represented by the
36 letters. For this reason the need rises for \textit{diacritics}, which are symbols ``decorating''
original letters. Usually, a \textit{diacritic} is written above or under the letter to emphasize the
short vowel accompanied with that letter. There are 4 diacritics: \mbox{\textarabic{◌َ} \textarabic{◌ُ}
  \textarabic{◌ِ} \textarabic{◌ْ}}. Table~\ref{arabic:diacritics_dal} lists these 4 diacritics on an
example letter \textarabic{د}, their transliterated names, along with their short vowel
representation. Each of the three diacritics \mbox{\textarabic{◌َ} \textarabic{◌ُ} \textarabic{◌ِ}} is called \textit{harakah}; whereas the fourth \textarabic{◌ْ} is called
\textit{sukun}. Diacritics are just to make short vowels clearer; however, their writing is not
compulsory since they can be almost inferred from the grammatical rules and the semantic of the
text. Moreover, a phrase with diacritics written for only some letters is linguistically sound.

There are two more sub-diacritics made up of the basic four. The first is known as \textit{shaddah}
\textarabic{◌ّ}, which must associate with one of the three \textit{harakah} and written as
\mbox{\textarabic{◌َّ} \textarabic{◌ُّ} \textarabic{◌ِّ}}. \textit{Shaddah} is a shorthand writing for the
case when a letter appears two times in a row where the first occurrence is accompanied with
\textit{sukun} and the second occurrence is accompanied with \textit{harakah}. Then, for short, it
is written as one occurrence accompanied with \textit{shaddah} associated with the corresponding
\textit{harakah}. E.g., \mbox{\textarabic{دْدَ}} is written as \textarabic{دَّ}. The second is known as
\textit{tanween}, which must associate as well one of the three \textit{harakah} and written as:
\mbox{\textarabic{◌ٍ} \textarabic{◌ٌ} \textarabic{◌ً}}. \textit{Tanween} accompanies the last letter of
some words, according to Arabic grammar, ending with \textit{harakah}. This is merely for reminding
the reader to pronounce the word as if there is \textarabic{نْ} (sounding as \textit{/n/}), follows
that \textit{harakah}. However, it is just a phone and is not a part of the word itself. E.g.,
\textarabic{رَجُلٌ} is pronounced \mbox{\textarabic{رَجُلُ + نْ}} and \textarabic{رَجُلٍ} is pronounced
\mbox{\textarabic{رَجُلِ + نْ}}.


\subsection{Arabic Poetry (\textarabic{الشعر العربى})}\label{sec:arab-poetry-text}
Arabic poetry is the earliest form of Arabic literature; it dates back to the sixth century. Poets
wrote poems without knowing exactly what rules make a collection of words a poem. People recognize
poetry by nature, but only talented ones who could write poems. This was the case until
\textit{Al-Farahidi} (718 – 786 CE) has analyzed Arabic poems and recognized their patterns. He came
up with that the succession of consonants and vowels, and hence \textit{harakah} and \textit{sukun},
rather than the succession of letters themselves, produces patterns (\textit{meters}) which keeps
the balanced music of pieces of poem. He recognized fifteen meters. Later, one of his students,
\textit{Al-khfash}, discovered one more meter to make them all sixteen. Arabs call meters
\textarabic{بحور}, which means ``seas''~\cite{Moustafa}.

A poem is a collection of verses. A verse example is:%
\begin{Arabic}
  \begin{traditionalpoem*}
    قفا نبك من ذِكرى حبيب ومنزل\quad & \quad بسِقطِ اللِّوى بينَ الدَّخول فحَوْملِ
  \end{traditionalpoem*}
\end{Arabic}
A verse, known in Arabic as \textit{bayt} \textarabic{(بَيت)}, which consists of two halves. Each
half is called a \textit{shatr} (\textarabic{شطر}).  \textit{Al-Farahidi} has introduced
\textit{al-'arud} (\textarabic{العروض}), which is often called the \textit{Knowledge of Poetry} or
the study of poetic meters. He laid down rigorous rules and measures, with them we can determine
whether a meter of a poem is sound or broken. For the present article to be fairly self-contained,
where many details are reduced, a very brief introduction to \textit{al-'arud} is provided through
the following lines.
\begin{table}[!tb]
  \centering
  \resizebox{\columnwidth}{!}{%
    \begin{tabular}[t!]{ccccccccc}
      \toprule
      \textbf{Foot}& \textarabic{فَعُوْلُنْ} & \textarabic{فَاْعِلُنْ} & \textarabic{مُسْتَفْعِلُنْ} & \textarabic{مَفَاْعِيْلُنْ} &\textarabic{مَفْعُوْلَاْتُ} &\textarabic{فَاْعِلَاْتُنْ} &\textarabic{مُفَاْعَلَتُنْ} &\textarabic{مُتَفَاْعِلُنْ}\\
      \midrule
      \textbf{Scansion}&\texttt{0/0//}&\texttt{0//0/}&\texttt{0//0/0/}&\texttt{0/0/0//}&\texttt{/0/0/0/}&\texttt{0/0//0/}&\texttt{0///0//}&\texttt{0//0///}\\
      \bottomrule
    \end{tabular}}
  \caption{The eight feet of Arabic poetry. Every digit (\texttt{/} or \texttt{0}) represents the
    corresponding diacritic over a letter of a feet: \textit{harakah} (\mbox{\textarabic{◌َ}
      \textarabic{◌ُ} \textarabic{◌ِ}}) or \textit{sukun} (\textarabic{◌ْ}) respectively. Any of the
    three letters \mbox{\textarabic{و ا ى}} (called \textit{mad}) is equivalent to \texttt{0};
    \textit{tanween} and \textit{shaddah} are equivalent to \texttt{0/} and \texttt{/0}
    respectively.}\label{arud:feet}
\end{table}

A meter is an ordered sequence of phonetic syllables (blocks or mnemonics) called \textit{feet}. A
foot is written with letters only having \textit{harakah} or \textit{sukun}, i.e., with neither
\textit{shaddah} nor \textit{tanween}; and hence each letter in a foot maps directly to either a
consonant or a vowel. Therefore, feet represent phonetic mnemonics, of the pronounced
poem, called \textit{tafa'il} (\textarabic{تفاعيل}). Table~\ref{arud:feet} lists the eight feet used
by Arabs and the pattern (scansion) of \textit{harakah} and \textit{sukun} of each foot, where a
\textit{harakah} is represented as \texttt{/} and a \textit{sukun} is represented as
\texttt{0}. Each scansion reads RTL to match the letters of the corresponding foot.

According to \textit{Al-Farahidi} and his student, they discovered sixteen combinations of
\textit{tafa'il} in Arabic poems; they called each combination a \textit{meter}
(\textarabic{بحر}). (Theoretically speaking, there is no limit for either the number of
\textit{tafa'il} or their combinations; however, Arab composed poems using only this structure). A
meter appears in a \textit{verse} twice, once in each \textit{shatr}. E.g., \mbox{\textarabic{وَيُسْأَلُ
    فِي الحَوادِثِ ذو صَواب}} is the first \textit{shatr} of a verse of \textit{Al-Wafeer} meter
\mbox{\textarabic{مُفَاْعَلَتُنْ مُفَاْعَلَتُنْ فَعُوْلُنْ}}. The pattern of the \textit{harakah} and \textit{sukun} of
this meter is \mbox{\texttt{0/0// 0///0// 0///0//}} (RTL), and is obtainable directly by replacing
each of the three feet by the corresponding code in table~\ref{arud:feet}. This pattern corresponds
exactly to the pattern of \textit{harakah} and \textit{sukun} of the pronounced (not written)
\textit{shatr}. E.g., the pronunciation of the first two words and the first two letters of the
third word \mbox{\textarabic{{\color{OliveGreen}وَيُسْأَلُ} {\color{fgred}فِي الْـ}}} has exactly the same
pattern as the first of the three \textit{{tafa'il}} of the meter
\mbox{\textarabic{{\color{OliveGreen}مُفَاعَلَـ}{\color{fgred}تُنْ}}}, and both have the scansion
\texttt{\color{fgred}{0/}\color{OliveGreen}{//0//}}. For more clarification, the colored parts have
corresponding pronunciation pattern; which emphasizes that the start and end of a word do not have
to coincide with the start and end of the phonetic syllable. The pronunciation of the rest of the
\textit{shatr} \mbox{\textarabic{حوادث ذو صواب}} maps to the rest of the meter
\mbox{\textarabic{مفاعلتن فعولن}}. Any other poem, regardless of its wording and semantic, following
the same meter, i.e., following the same pattern of \textit{harakah} and \textit{sukun}, will have
the same pronunciation or phonetic pattern.
\begin{table}[!tb]
  \centering
  \resizebox{\columnwidth}{!}{%
    \begin{tabular}[t!]{lrr}
      \toprule
      \textbf{Meter Name}   & \textbf{Pattern} & \textbf{Scansion}\\
      \midrule
      \textit{al-Taweel}    & \textarabic{فَعُوْلُنْ مَفَاْعِيْلُنْ فَعُوْلُنْ مَفَاْعِلُنْ} & \texttt{0//0// ~~0/0// 0/0/0// ~~0/0//}  \\
      \textit{al-Kamel}     & \textarabic{مُتَفَاْعِلُنْ مُتَفَاْعِلُنْ مُتَفَاْعِلُنْ}     & \texttt{0//0/// 0//0/// 0//0///}\\
      \textit{al-Baseet}    & \textarabic{مُسْتَفْعِلُنْ فَاْعِلُنْ مُسْتَفْعِلُنْ فَاْعِلُنْ} & \texttt{0//0/ 0//0/0/ ~~0//0/ 0//0/0/} \\
      \textit{al-Khafeef}   & \textarabic{فَاْعِلَاْتُنْ مُسْتَفْعِلُنْ فَاْعِلَاْتُنْ}     & \texttt{0/0//0/ 0//0/0/ 0/0//0/}\\
      \textit{al-Wafeer}    & \textarabic{مُفَاْعَلَتُنْ مُفَاْعَلَتُنْ فَعُوْلُنْ}      & \texttt{0/0// 0///0// 0///0//}\\
      \textit{al-Rigz}      & \textarabic{مُسْتَفْعِلُنْ مُسْتَفْعِلُنْ مُسْتَفْعِلُنْ}     & \texttt{0//0/0/ 0//0/0/ 0//0/0/}\\
      \textit{al-Raml}      & \textarabic{فَاْعِلَاْتُنْ فَاْعِلَاْتُنْ فَاْعِلَاْتُنْ}     & \texttt{0/0//0/ 0/0//0/ 0/0//0/}\\
      \textit{al-Motakarib} & \textarabic{فَعُوْلُنْ فَعُوْلُنْ فَعُوْلُنْ فَعُوْلُنْ}     & \texttt{0/0// ~~0/0// ~~0/0// ~~0/0//}\\
      \textit{al-Sar'e}     & \textarabic{مُسْتَفْعِلُنْ مُسْتَفْعِلُنْ مَفْعُوْلَاْتُ}     & \texttt{/0/0/0/ 0//0/0/ 0//0/0/}\\
      \textit{al-Monsareh}  & \textarabic{مُسْتَفْعِلُنْ  مَفْعُوْلَاْتُ مُسْتَفْعِلُنْ}    & \texttt{0//0/0/ /0/0/0/ 0//0/0/}\\
      \textit{al-Mogtath}   & \textarabic{مُسْتَفْعِلُنْ فَاْعِلَاْتُنْ  فَاْعِلَاْتُنْ}    & \texttt{0/0//0/ 0/0//0/ 0//0/0/}\\
      \textit{al-Madeed}    & \textarabic{فَاْعِلَاْتُنْ فَاْعِلُنْ فَاْعِلَاْتُنْ}       & \texttt{0/0//0/ ~~0//0/ 0/0//0/}\\
      \textit{al-Hazg}      & \textarabic{مَفَاْعِيْلُنْ مَفَاْعِيْلُنْ}             & \texttt{0/0/0// 0/0/0//}\\
      \textit{al-Motadarik} & \textarabic{فَاْعِلُنْ فَاْعِلُنْ فَاْعِلُنْ فَاْعِلُنْ}     & \texttt{0//0/ ~~0//0/ ~~0//0/ ~~0//0/}\\
      \textit{al-Moktadib}  & \textarabic{مَفْعُوْلَاْتُ مُسْتَفْعِلُنْ مُسْتَفْعِلُنْ}     & \texttt{0//0/0/ 0//0/0/ /0/0/0/}\\
      \textit{al-Modar'e}   & \textarabic{مَفَاْعِيْلُنْ فَاْعِلَاْتُنْ فَاْعِلَاْتُنْ}     & \texttt{0/0//0/ 0/0//0/ 0/0/0//}\\
      \bottomrule
    \end{tabular}}
  \caption{The sixteen meters of Arabic poem: transliterated names (1st col.), mnemonics or
    \textit{tafa'il} (2nd col.), and the corresponding pattern of \textit{harakah} and
    \textit{sukun} in \texttt{0/} format or scansion (3rd col.).}\label{arud:meters}
\end{table}

Table~\ref{arud:meters} lists the names of all the sixteen meters, the transliteration of their
names, and their patterns (scansion). Each pattern is written in two equivalent forms: the feet
style using the eight feet of Table~\ref{arud:feet} and the scansion pattern using the \texttt{0/}
symbols. The scansion is written in groups; each corresponds to one foot and all are RTL\@.


\subsection{English poetry}\label{sec:english-poetry}
English poetry dates back to the seventh century. At that time poems were written in
\textit{Anglo-Saxon}, also known as \textit{Old English}. Many political changes have influenced
the language until it became as it is nowadays. English prosody was not formalized rigorously as a
stand-alone knowledge, but many tools of the \textit{Greek} prosody were borrowed to describe it.

A \textit{syllable} is the unit of pronunciation having one vowel, with or without surrounding
consonants. English words consist of one or more syllables. For example the word \mbox{``water''}
(pronounced as \mbox{\textipa{\sffamily /"wO:t@(r)/}}) consists of two phonetic syllables:
\mbox{\textipa{\sffamily /"wO:/}} and \mbox{\textipa{\sffamily /t@(r)/}}. Each syllable has only one
vowel sound. Syllables can be either stressed or unstressed and will be denoted by \textit{/} and
$\times$ respectively. In phonology, a stress is a phonetic emphasis given to a syllable, which can
be caused by, e.g., increasing the loudness, stretching vowel length, or changing the sound
pitch. In the previous ``water'' example, the first syllable is stressed, which means it is
pronounced with high sound pitch; whereas the second syllable is unstressed which means it is
pronounced in low sound pitch. Therefore, ``water'' is a stressed-unstressed word, which can be
denoted by \mbox{\textit{/}$\times$}. Stresses are shown in the phonetic script using the primary stress
symbol \textipa{\sffamily (")}.
\begin{table}[!tb]
  \centering
  \resizebox{\columnwidth}{!}{%
    \begin{tabular}{cccccccc}
      \toprule
      \textbf{{Foot}} & \textit{Iamb} & \textit{Trochee} & \textit{Dactyl} & \textit{Anapest} & \textit{Pyrrhic} & \textit{Amphibrach} & \textit{Spondee} \\
      \midrule
      \textbf{Stresses}& $\times$\textit{/} & \textit{/}$\times$ & \textit{/}$\times\times$ & $\times\times$\textit{/} & $\times\times$ & $\times$\textit{/}$\times$ & \textit{/}\textit{/} \\
      \bottomrule
    \end{tabular}}
  \caption{The seven feet of English poem. Every foot is a combination of stressed and unstressed
    syllables, denoted by \textit{/} and \textit{x} respectively.}\label{feet}
\end{table}
There are seven different combinations of stressed and unstressed syllables that make the seven
poetic \textit{feet}.  They are shown in table~\ref{feet}. Meters are described as a sequence of
feet. English meters are \textit{qualitative} meters, which are stressed syllables coming at regular
intervals. A meter is defined as the repetition of one of the previous seven feet one to eight
times. If the foot is repeated once, then the verse is \textit{monometer}, if it is repeated twice
then it is a \textit{dimeter} verse, and so on until \textit{octameter} which means a foot is
repeated eight times. This is an example, where stressed syllables are bold: \mbox{``That
  \textbf{time} of \textbf{year} thou \textbf{mayst} in \textbf{me} be\textbf{hold}''}. The previous
verse belongs to the \textit{Iamb} meter, with the pattern \mbox{$\times$\textit{/}} repeated five times;
so it is an \textit{Iambic pentameter} verse.

\subsection{Paper Organization}\label{sec:paper-organization}
The rest of this paper is organized as follows. Sec.~\ref{sec:literature-review} is a literature
review of meter detection of both languages; the novelty of our approach and the point of departure
from the literature will be emphasized. Sec.~\ref{sec:datasets} explains the data acquisition steps
and the data repository created by this project to be publicly available for future research; in
addition, this section explains character encoding methods, along with our new encoding method and
how they are applied to Arabic letters in particular. Sec.~\ref{sec:model} explains how experiments
are designed and conducted in this research. Sec.~\ref{sec:results} presents and interprets the
results of these experiments. Sec.~\ref{sec:discussion} is a discussion, where we emphasize the
interpretation of some counter-intuitive results and connect them to the size of conducted
experiments, and the remedy in the future work that is currently under implementation.
