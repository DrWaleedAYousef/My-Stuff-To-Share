\section{Literature review}\label{sec:literature-review}
To the best of our knowledge, the problem addressed in the present paper has never been addressed in
the literature. ``Learning'' poem style from text so that machines are able to classify unseen
written poem to the right meter seems to be a novel area. However, there is some literature on
recognizing the meters of written Arabic poem using rule-based deterministic algorithms. We did not
find related work on English written poem. These rules are derived by humans/experts and not learned
by machines from data. In this regard, this is quite irrelevant to our present problem, and this is
our point of departure in this research. However, we review these methods for the sake of
completion.

\cite{Kurt2012AlgorithmForDetectionAnalysis} worked on the Ottoman Language. They converted the
Ottoman text into a lingual form; in particular, the poem was transliterated to Latin transcription
alphabet (LTA). Next, the text was fed to the algorithm, which uses a database containing all
Ottoman meters, to be compared to the existing meters and then classified to the closest one.

\cite{Alnagdawi2013FindingArabicPoemMeter} worked on Arabic language. They formalized the
\textit{scansion}s, \textit{al-'arud}, and some lingual rules (like pronounced and silent rules,
which are directly related to \textit{harakah} and \textit{sukun}) in terms of context-free grammar
and regular expression templates. The classification accuracy was only 75\% on a very small sample
of 128 verses.

\cite{Abuata2016RuleBasedAlgorithmFor} worked on Arabic language. They designed a five-step
deterministic \textit{algorithm} for analyzing and detecting meters. First, they input text carrying
full diacritics for all letters. Second, they convert the input text into \textit{al-'arud} writing
style (Sec.~\ref{sec:arab-poetry-text}) using \texttt{if-else} rules. Third, the metrical
\textit{scansion} rules are applied, which leaves the input text as a sequence of zeros and
ones. Fourth, each group of zeros and ones are defined as a \textit{tafa'il}
(Table~\ref{arud:feet}). Finally, the input text is classified to the closest meter to the
\textit{tafa'il} sequence (Table~\ref{arud:meters}). The classification accuracy of this algorithm
is 82.2\%, on a relatively small sample of 417 verses.

\bigskip

It is quite important to observe that although these algorithms are deterministic rules that are fed
by experts, alas, they did not succeed in producing high accuracy, 75\% and 82.2\%. This is in
contrast to our featureless RNN approach that remarkably outperforms these methods by achieving
96.38\%. The interpretation of that is clear. The rule-based algorithms cannot list all possible
combinations of anomalies in written text, including missing diacritics on some characters, breaking
the meter by a poet, etc; whereas, RNN will be able to ``learn'' by example the probability of these
occurrences. Table~\ref{tab:summ-results} summarizes the accuracies and the testing sample size of
this literature in comparison with our approach. It is even more surprising that while these
algorithms must work on poem with diacritics, RNN accuracy only dropped about 1\% when trained on
plain poem with no diacritics.

\begin{table}[!tb]
  \centering
  \begin{tabular}{c c c c}
    \toprule
    \textbf{Ref.}& \textbf{Accuracy}& \textbf{Test Size} & \textbf{Poem}\\
    \midrule
    \cite{Alnagdawi2013FindingArabicPoemMeter}   & 75\%     & 128     & \multirow{3}{*}{Arabic}\\
    \cite{Abuata2016RuleBasedAlgorithmFor}      & 82.2\%   & 417     & \\
    This article   & 96.38\%  & 150,000 & \\
    \midrule
    This article   & 82.31\%  &  1,740  & English\\
    \bottomrule
  \end{tabular}
  \caption{Overall accuracy of this article compared to literature.}\label{tab:summ-results}
\end{table}
